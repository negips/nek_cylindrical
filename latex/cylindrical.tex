\documentclass{kthreport}

\usepackage{color}
\usepackage{natbib}
\usepackage[shortlabels]{enumitem}
\usepackage[margin=5pt,skip=0pt]{caption}
\usepackage{placeins}
\usepackage{bibentry}
\usepackage{amsmath}
\usepackage{cancel}
\usepackage{tabularx}
\usepackage{amssymb}
\usepackage{bm}			% Bold math
\usepackage{tabstackengine}
\usepackage{xcolor}
\usepackage{relsize}
%\usepackage{biblatex}
%\addbibresource{licentiate.bib}
%\DeclareBibliographyCategory{ignore}
%\addtocategory{ignore}{poirel08,poirel10,barnes16,yuan13,barnes18}

% default language is English, but you can use Swedish definitions like this:
% \documentclass[swedish]{kthreport}

% Remember that in order for the class to find the KTH logo, you need to
% have it in your path, for example:
% export TEXINPUTS=/path/to/logo/location//:$TEXINPUTS

\title{Navier--Stokes in Cylindrical formulation}
\subtitle{Weak form and Nek implementation}
\author{Prabal Negi}
%\diarienr{99999-99}
\newcommand{\ignorecite}[1]{{\@fileswfalse\cite{#1}}}
\newcommand{\ignorecitep}[1]{{\@fileswfalse\citep{#1}}}

\newcommand{\bigcdot}{\bm{\cdot}}

\begin{document}
\maketitle

\section{Formulation of Navier--Stokes}

\subsection{Mapping from Cartesian to Cylindrical}

\begin{subequations}
	\begin{eqnarray}	
		x_{c}	&= & x \\
		y_{c}   &= & r\cos(\theta)\\
		z_{c}   &= &r\sin(\theta)
			\label{eqn:coord_mapping}
	\end{eqnarray}
\end{subequations}

%\begin{subequations}
%	\begin{eqnarray}	
%		x			&= & x_{c} \\
%		r   		&= & \sqrt{y_{c}^2 + z_{c}^2}\\
%		\theta &= & \arctan(y_{c},z_{c})
%		\label{eqn:coord_mapping2}
%	\end{eqnarray}
%\end{subequations}


	\begin{eqnarray}
		\begin{Bmatrix}
			dx_{c} \\
			dy_{c} \\
			dz_{c}
		\end{Bmatrix}
		 = 
		\begin{pmatrix}
		    1 &  0 &  0 \\
			0 & \cos(\theta) & -r\sin(\theta) \\
			0 & \sin(\theta) & r\cos(\theta)  
		\end{pmatrix}
		\begin{Bmatrix}
			dx \\
			dr \\
			d\theta
		\end{Bmatrix} \\
	\implies
	\begin{Bmatrix}
		dx \\
		dr \\
		d\theta
	\end{Bmatrix}
	= 
	\begin{pmatrix}
		1 &  0 &  0 \\
		0 & \cos(\theta) & \sin(\theta) \\
		0 & -\sin(\theta)/r & \cos(\theta)/r  
	\end{pmatrix}
	\begin{Bmatrix}
	dx_{c} \\
	dy_{c} \\
	dz_{c}
\end{Bmatrix}
	\end{eqnarray}

\begin{subequations}
	\begin{eqnarray}	
		\hat{x}	&= & \hat{x_{c}} \\
		\hat{r}   &= & \cos(\theta)\hat{y_{c}} + \sin(\theta)\hat{z_{c}}\\
		\hat{\theta}   &= & -\sin(\theta)\hat{y_{c}} + \cos(\theta)\hat{z_{c}}
		\label{eqn:unit_vector_mapping}
	\end{eqnarray}
\end{subequations}

\begin{eqnarray}
\implies
	\begin{Bmatrix}
		d\hat{x} \\
		d\hat{r}\\
		d\hat{\theta}
	\end{Bmatrix}
	&=& 
	\begin{pmatrix}
		0 &  0 &  0 \\
		0 & 0 &  -\sin(\theta)\hat{y_{c}} + \cos(\theta)\hat{z_{c}} \\
		0 & 0 & -\cos(\theta)\hat{y_{c}} - \sin(\theta)\hat{z_{c}}  
	\end{pmatrix}
	\begin{Bmatrix}
		dx \\
		dr \\
		d\theta
	\end{Bmatrix} \nonumber \\
%
\implies 	
\begin{Bmatrix}
	d\hat{x} \\
	d\hat{r}\\
	d\hat{\theta}
\end{Bmatrix}
& = &
\begin{pmatrix}
	0 &  0 &  0 \\
	0 & 0 &  \hat{\theta} \\
	0 & 0 & -\hat{r} 
\end{pmatrix}
\begin{Bmatrix}
	dx \\
	dr \\
	d\theta
\end{Bmatrix}
\label{eqn:Jacobian_unit_vector}
\end{eqnarray}

\subsection{Gradient/Divergence/Laplace Operators}

For scalar fields
\begin{eqnarray}
	\begin{split}
	\nabla(\psi) 	= & \left(\frac{\partial\psi}{\partial x}\frac{\partial x}{\partial x_{c}} + \frac{\partial\psi}{\partial r}\frac{\partial r}{\partial x_{c}} + \frac{\partial\psi}{\partial \theta}\frac{\partial \theta}{\partial x_{c}}\right)\hat{x_{c}} + \\
	%
	& \left(\frac{\partial\psi}{\partial x}\frac{\partial x}{\partial y_{c}} + \frac{\partial\psi}{\partial r}\frac{\partial r}{\partial y_{c}} + \frac{\partial\psi}{\partial \theta}\frac{\partial \theta}{\partial y_{c}}\right)\hat{y_{c}} + \\
	%
	& \left(\frac{\partial\psi}{\partial x}\frac{\partial x}{\partial z_{c}} + \frac{\partial\psi}{\partial r}\frac{\partial r}{\partial z_{c}} + \frac{\partial\psi}{\partial \theta}\frac{\partial \theta}{\partial z_{c}} \right)\hat{z_{c}}
	\end{split} \nonumber
\end{eqnarray}

\begin{eqnarray}
	\therefore \nabla(\psi) 	= \frac{\partial\psi}{\partial x}\hat{x} + \frac{\partial\psi}{\partial r}\hat{r} + \frac{1}{r}\frac{\partial\psi}{\partial \theta}\hat{\theta}
	\label{eqn:scalar_gradient}
\end{eqnarray}

Laplacian of a scalar field
\begin{eqnarray}
\nabla\cdot\nabla(\psi) = \left[\frac{\partial}{\partial x}\hat{x} + \frac{\partial}{\partial r}\hat{r} + \frac{1}{r}\frac{\partial}{\partial \theta}\hat{\theta} \right]\cdot\left[\hat{x}\frac{\partial\psi}{\partial x} + \hat{r}\frac{\partial\psi}{\partial r} + \hat{\theta}\frac{1}{r}\frac{\partial\psi}{\partial \theta}\right] \nonumber \\
= \left\{
\begin{split}
	 \left[\hat{x}\cdot\hat{x}\frac{\partial }{\partial x}\frac{\partial \psi}{\partial x} + \frac{\partial \psi}{\partial x}\frac{\partial \hat{x}}{\partial x}\cdot\hat{x} \right] 
	 + \left[\hat{x}\cdot\hat{r} \frac{\partial }{\partial x}\frac{\partial \psi}{\partial r} + \frac{\partial \psi}{\partial r}\frac{\partial \hat{r}}{\partial x}\cdot\hat{x}		\right] 
	 + \left[\hat{x}\cdot\hat{\theta} \frac{\partial }{\partial x}\left(\frac{1}{r}\frac{\partial \psi}{\partial \theta}\right) + \frac{1}{r}\frac{\partial \psi}{\partial \theta}\frac{\partial \hat{\theta}}{\partial x}\cdot\hat{x}		\right] \\
	 %
	 + \left[\hat{r}\cdot\hat{x}\frac{\partial }{\partial r}\frac{\partial \psi}{\partial x} +   \frac{\partial \psi}{\partial x}\frac{\partial \hat{x}}{\partial r}\cdot\hat{r} \right] 
	 +  \left[\hat{r}\cdot\hat{r} \frac{\partial }{\partial r}\frac{\partial \psi}{\partial r} + \frac{\partial \psi}{\partial r}\frac{\partial \hat{r}}{\partial r}\cdot\hat{r}		\right] 
	 + \left[\hat{r}\cdot\hat{\theta} \frac{\partial }{\partial r}\left(\frac{1}{r}\frac{\partial \psi}{\partial \theta}\right) + \frac{1}{r}\frac{\partial \psi}{\partial \theta}\frac{\partial \hat{\theta}}{\partial r}\cdot\hat{r}		\right] \\
	 %
	 + \left[\hat{\theta}\cdot\hat{x}\frac{1}{r}\frac{\partial }{\partial \theta}\frac{\partial \psi}{\partial x} + \frac{1}{r}\frac{\partial \psi}{\partial x}\frac{\partial \hat{x}}{\partial \theta}\cdot\hat{\theta} \right]
	 +  \left[\hat{\theta}\cdot\hat{r}\frac{1}{r}\frac{\partial }{\partial \theta}\frac{\partial \psi}{\partial r} + \frac{1}{r}\frac{\partial \psi}{\partial r}\frac{\partial \hat{r}}{\partial \theta}\cdot\hat{\theta}		\right] \\
	 + \left[\hat{\theta}\cdot\hat{\theta} \frac{1}{r}\frac{\partial }{\partial \theta}\left(\frac{1}{r}\frac{\partial \psi}{\partial \theta}\right) + \frac{1}{r^2}\frac{\partial \psi}{\partial \theta}\frac{\partial \hat{\theta}}{\partial \theta}\cdot\hat{\theta}		\right]
\end{split}\right. \nonumber
\end{eqnarray}
%
\begin{eqnarray}
\therefore \nabla^{2}(\psi) = \frac{\partial^{2}\psi}{\partial x^{2}} +\frac{1}{r}\frac{\partial }{\partial r}\left(r\frac{\partial \psi}{\partial r}	\right) + \frac{1}{r^{2}}\frac{\partial^{2}\psi}{\partial \theta}
	\label{eqn:scalar_laplacian}
\end{eqnarray}

Divergence of a Vector field
\begin{eqnarray}
	\nabla\cdot(\bm{u}) = \left(\hat{x}\cdot\dfrac{\partial }{\partial x} + \hat{r}\cdot\dfrac{\partial }{\partial r} + \hat{\theta}\cdot\dfrac{1}{r}\dfrac{\partial }{\partial \theta}	\right) \left(u_{x}\hat{x} + u_{r}\hat{r} + u_{\theta}\hat{\theta} \right) \nonumber \\
	\begin{split}
	\nabla\cdot(\bm{u}) = & \quad \left(\hat{x}\cdot\hat{x}\dfrac{\partial u_{x}}{\partial x} + 
	\hat{x}\cdot\dfrac{\partial \hat{x}}{\partial x}u_{x} + \hat{x}\cdot\hat{r}\dfrac{\partial u_{r}}{\partial x} + \hat{x}\cdot\dfrac{\partial\hat{r}}{\partial x}u_{r} + 
	\hat{x}\cdot\hat{\theta}\dfrac{\partial u_{\theta}}{\partial x} +
	\hat{x}\cdot\dfrac{\partial \hat{\theta}}{\partial x}u_{\theta} \right) \\
	%
	& + \left(\hat{r}\cdot\hat{x}\dfrac{\partial u_{x}}{\partial r} + 		\hat{r}\cdot\dfrac{\partial \hat{x}}{\partial r}u_{x} + \hat{r}\cdot\hat{r}\dfrac{\partial u_{r}}{\partial r} + \hat{r}\cdot\dfrac{\partial\hat{r}}{\partial r}u_{r} + 
	\hat{r}\cdot\hat{\theta}\dfrac{\partial u_{\theta}}{\partial r} +
	\hat{r}\cdot\dfrac{\partial \hat{\theta}}{\partial r}u_{\theta} \right) \\
	%
	& + \left(\hat{\theta}\cdot\hat{x}\dfrac{1}{r}\dfrac{\partial u_{x}}{\partial \theta} +	
	\hat{\theta}\cdot\dfrac{\partial \hat{x}}{\partial \theta}\dfrac{1}{r}u_{x} +
	\hat{\theta}\cdot\hat{r}\dfrac{1}{r}\dfrac{\partial u_{r}}{\partial \theta} +
	\hat{\theta}\cdot\dfrac{\partial\hat{r}}{\partial \theta}\dfrac{1}{r}u_{r} + 
	\hat{\theta}\cdot\hat{\theta}\dfrac{1}{r}\dfrac{\partial u_{\theta}}{\partial \theta} +
	\hat{\theta}\cdot\dfrac{\partial \hat{\theta}}{\partial \theta}\dfrac{1}{r}u_{\theta} \right)	
	\end{split} \nonumber
\end{eqnarray}

\begin{eqnarray}
	\begin{split}
		\nabla\cdot(\bm{u}) = & \quad \left(\dfrac{\partial u_{x}}{\partial x} + \dfrac{\partial u_{r}}{\partial r} + \dfrac{u_{r}}{r} + 
		\dfrac{1}{r}\dfrac{\partial u_{\theta}}{\partial \theta}
		\right)	
	\end{split} \label{eqn:cylindrical_divergence}
\end{eqnarray}

Vector gradient. Highlighted in color are the only terms which are non-zero differentials of unit vectors.
\begin{eqnarray}
	\nabla(\bm{u}) = \left[\hat{x}\frac{\partial}{\partial x} + \hat{r}\frac{\partial}{\partial r} + \hat{\theta}\frac{1}{r}\frac{\partial}{\partial \theta} \right]\left[u_{x}\hat{x} + u_{r}\hat{r} + u_{\theta}\hat{\theta}\right] \nonumber \\
	= \left\{\begin{split}
	\left[\hat{x}\frac{\partial u_{x}}{\partial x} \hat{x} + \hat{x}u_{x}\frac{\partial \hat{x}}{\partial x} \right] 
	+ \left[\hat{x}\frac{\partial u_{r}}{\partial x}\hat{r} + \hat{x}u_{r}\frac{\partial \hat{r}}{\partial x} \right] 
	+ \left[\hat{x}\frac{\partial u_{\theta}}{\partial x}\hat{\theta}	+ \hat{x}u_{\theta}\frac{\partial \hat{\theta}}{\partial x}\right] \\
	%
	+ \left[\hat{r}\frac{\partial u_{x}}{\partial r}\hat{x} + \hat{r}u_{x}\frac{\partial \hat{x}}{\partial r} \right] 
	+ \left[\hat{r}\frac{\partial u_{r}}{\partial r}\hat{r} + \hat{r}u_{r}\frac{\partial \hat{r}}{\partial r} \right] 
	+ \left[\hat{r}\frac{\partial u_{\theta}}{\partial r}\hat{\theta} + \hat{r}u_{\theta}\frac{\partial \hat{\theta}}{\partial r}\right] \\
	%
	+ \left[\hat{\theta}\frac{1}{r}\frac{\partial u_{x}}{\partial \theta}\hat{x} + 	\hat{\theta}\frac{u_{x}}{r}\frac{\partial \hat{x}}{\partial \theta}\right]
	+ \left[\hat{\theta}\frac{1}{r}\frac{\partial u_{r}}{\partial \theta}\hat{r} + 	\hat{\theta}\color{red}{\frac{u_{r}}{r}\frac{\partial \hat{r}}{\partial \theta}} \color{black} \right]
	+ \left[\hat{\theta}\frac{1}{r}\frac{\partial u_{\theta}}{\partial \theta}\hat{\theta} +  \hat{\theta}\color{blue}{\frac{u_{\theta}}{r}\frac{\partial \hat{\theta}}{\partial \theta}} \color{black}\right]
	\end{split}\right. \\
%
= \left\{\begin{split}
	\left[\hat{x}\frac{\partial u_{x}}{\partial x} \hat{x} + \hat{x}\frac{\partial u_{r}}{\partial x}\hat{r} + \hat{x}\frac{\partial u_{\theta}}{\partial x}\hat{\theta}	\right] \\
	%
	+ \left[\hat{r}\frac{\partial u_{x}}{\partial r}\hat{x} + \hat{r}\frac{\partial u_{r}}{\partial r}\hat{r} + \hat{r}\frac{\partial u_{\theta}}{\partial r}\hat{\theta} \right] \\
	%
	+ \left[\hat{\theta}\frac{1}{r}\frac{\partial u_{x}}{\partial \theta}\hat{x} + 	 \hat{\theta}\frac{1}{r}\frac{\partial u_{r}}{\partial \theta}\hat{r} + 	\hat{\theta}\color{red}{\frac{u_{r}}{r}\hat{\theta}} \color{black} 
	+ \hat{\theta}\frac{1}{r}\frac{\partial u_{\theta}}{\partial \theta}\hat{\theta} -  \hat{\theta}\color{blue}{\frac{u_{\theta}}{r}\hat{r}} \color{black}\right]
\end{split}\right.
\end{eqnarray}

Which, I rearrange as gradients of individual velocity components (and their respective unit vectors)
\begin{subequations}
\begin{eqnarray}
\nabla (u_{x}\hat{x}) =& 
	\hat{x}\dfrac{\partial u_{x}}{\partial x} \hat{x} + \hat{r}\dfrac{\partial u_{x}}{\partial r}\hat{x} + \hat{\theta}\dfrac{1}{r}\dfrac{\partial u_{x}}{\partial \theta}\hat{x} \\
	%
	\nabla (u_{r}\hat{r}) =& \hat{x}\dfrac{\partial u_{r}}{\partial x} \hat{r} + \hat{r}\dfrac{\partial u_{r}}{\partial r}\hat{r} + \hat{\theta}\dfrac{1}{r}\dfrac{\partial u_{r}}{\partial \theta}\hat{r}  + \hat{\theta}\dfrac{u_{r}}{r}\hat{\theta}\\
	%
	\nabla (u_{\theta}\hat{\theta}) =& \hat{x}\dfrac{\partial u_{\theta}}{\partial x} \hat{\theta} + \hat{r}\dfrac{\partial u_{\theta}}{\partial r}\hat{\theta} + \hat{\theta}\dfrac{1}{r}\dfrac{\partial u_{\theta}}{\partial \theta}\hat{\theta}  - \hat{\theta}\dfrac{u_{\theta}}{r}\hat{r}
\end{eqnarray}
\end{subequations}


\begin{eqnarray}
	\def\arraystretch{2.5}
	\therefore \nabla(\bm{u}) = 
	\begin{bmatrix}
		\hat{x}\dfrac{\partial u_{x}}{\partial x}\hat{x} & \hat{x}\dfrac{\partial u_{r}}{\partial x}\hat{r}   &  \hat{x}\dfrac{\partial u_{\theta}}{\partial x}\hat{\theta} \\
		\hat{r}\dfrac{\partial u_{x}}{\partial r}\hat{x}  & \hat{r}\dfrac{\partial u_{r}}{\partial r}\hat{r}	&   \hat{r}\dfrac{\partial u_{\theta}}{\partial r}\hat{\theta}\\
		\hat{\theta}\dfrac{1}{r}\dfrac{\partial u_{x}}{\partial \theta}\hat{x} & \hat{\theta}\left(\color{blue}{-\dfrac{u_{\theta}}{r}} \color{black} + \dfrac{1}{r}\dfrac{\partial u_{r}}{\partial \theta} \right)\hat{r} 	& \hat{\theta}\left( \color{red}{\dfrac{u_{r}}{r}} \color{black} +\dfrac{1}{r} \dfrac{\partial u_{\theta}}{\partial \theta} \right)\hat{\theta}
	\end{bmatrix}
\end{eqnarray}

Vector Laplacian:
\begin{eqnarray}
	\nabla\cdot\nabla(\bm{u}) = \def\arraystretch{2.5} \begin{pmatrix}
		\hat{x}\cdot\dfrac{\partial}{\partial x} + \hat{r}\cdot\dfrac{\partial}{\partial r} + \hat{\theta}\cdot\dfrac{1}{r}\dfrac{\partial}{\partial \theta}
	\end{pmatrix}
	\begin{pmatrix}
	 \hat{x}\dfrac{\partial u_{x}}{\partial x}\hat{x} + & \hat{x}\dfrac{\partial u_{r}}{\partial x}\hat{r}  +  & \hat{x}\dfrac{\partial u_{\theta}}{\partial x}\hat{\theta} +\\
	 \hat{r}\dfrac{\partial u_{x}}{\partial r}\hat{x}  + &\hat{r}\dfrac{\partial u_{r}}{\partial r}\hat{r}   +  &\hat{r}\dfrac{\partial u_{\theta}}{\partial r}\hat{\theta} +\\
	 \hat{\theta}\dfrac{1}{r}\dfrac{\partial u_{x}}{\partial \theta}\hat{x} + &\hat{\theta}\left(-\dfrac{u_{\theta}}{r} \color{black} + \dfrac{1}{r}\dfrac{\partial u_{r}}{\partial \theta} \right)\hat{r} 	+ &\hat{\theta}\left( \dfrac{u_{r}}{r} +\dfrac{1}{r} \dfrac{\partial u_{\theta}}{\partial \theta} \right)\hat{\theta}
\end{pmatrix} \nonumber 
\end{eqnarray}

First Column
\begin{eqnarray}
		\left(\hat{x}\cdot\dfrac{\partial}{\partial x} + \hat{r}\cdot\dfrac{\partial}{\partial r} + \hat{\theta}\cdot\dfrac{1}{r}\dfrac{\partial}{\partial \theta}\right)
	\left(	\hat{x}\dfrac{\partial u_{x}}{\partial x}  + \hat{r}\dfrac{\partial u_{x}}{\partial r} + \hat{\theta}\dfrac{1}{r}\dfrac{\partial u_{x}}{\partial \theta}
	\right) \hat{x} \nonumber \\
	%
	\begin{split}
		= & \quad \hat{x}\cdot\left(\hat{x}\dfrac{\partial^{2}u_{x}}{\partial x^{2}} + \dfrac{\partial \hat{x}}{\partial x}\dfrac{\partial u_{x}}{\partial x}\right)\hat{x}  + \hat{x}\cdot\hat{x}\dfrac{\partial u_{x}}{\partial x}\dfrac{\partial \hat{x}}{\partial x}\\
		%
		&+ \hat{x}\cdot\left(\hat{r}\dfrac{\partial }{\partial x}\dfrac{\partial u_{x}}{\partial r} + \dfrac{\partial \hat{r}}{\partial x}\dfrac{\partial u_{x}}{\partial r}\right)\hat{x} + \hat{x}\cdot\hat{r}\dfrac{\partial u_{x}}{\partial r}\dfrac{\partial \hat{x}}{\partial r} \\
		%
		&+ \hat{x}\cdot\left(\hat{\theta}\dfrac{\partial }{\partial x}\left(\frac{1}{r}\dfrac{\partial u_{x}}{\partial \theta}\right) + \dfrac{\partial \hat{\theta}}{\partial x}\dfrac{1}{r}\dfrac{\partial u_{x}}{\partial \theta}				\right)\hat{x} + \hat{x}\cdot\hat{\theta}\dfrac{1}{r}\dfrac{\partial u_{x}}{\partial \theta}\dfrac{\partial \hat{x}}{\partial x}\\
		%
		&+ \hat{r}\cdot\left(\hat{x}\dfrac{\partial }{\partial r}\dfrac{\partial u_{x}}{\partial x}	+ \dfrac{\partial \hat{x}}{\partial r}\dfrac{\partial u_{x}}{\partial x}	\right)\hat{x} + \hat{r}\cdot\hat{x}\dfrac{\partial u_{x}}{\partial x}\dfrac{\partial \hat{x}}{\partial r}\\
		%
		&+ \hat{r}\cdot\left(\hat{r}\dfrac{\partial^{2}u_{x}}{\partial r^{2}} + \dfrac{\partial \hat{r}}{\partial r}\dfrac{\partial u_{x}}{\partial r}\right)\hat{x} + \hat{r}\cdot\hat{r}\dfrac{\partial u_{x}}{\partial r}\dfrac{\partial \hat{x}}{\partial r} \\
		%
		&+ \hat{r}\cdot\left(\hat{\theta}\dfrac{\partial }{\partial r}\left(\frac{1}{r}\dfrac{\partial u_{x}}{\partial \theta}\right) + \dfrac{\partial \hat{\theta}}{\partial r}\dfrac{1}{r}\dfrac{\partial u_{x}}{\partial \theta} \right)\hat{x} + \hat{r}\cdot\hat{\theta}\dfrac{1}{r}\dfrac{\partial u_{x}}{\partial \theta}\dfrac{\partial \hat{x}}{\partial r} \\
		%
		&+ \hat{\theta}\cdot\left(\hat{x}\dfrac{1}{r}\dfrac{\partial }{\partial \theta}\dfrac{\partial u_{x}}{\partial x}  + \dfrac{\partial \hat{x}}{\partial \theta}\dfrac{1}{r}\dfrac{\partial u_{x}}{\partial x} \right)\hat{x} + \hat{\theta}\cdot\hat{x}\dfrac{\partial u_{x}}{\partial x}\dfrac{1}{r}\dfrac{\partial \hat{x}}{\partial \theta} \\
		%
		&+ \hat{\theta}\cdot\left(	\hat{r}\dfrac{1}{r}\dfrac{\partial}{\partial \theta}\dfrac{\partial u_{x}}{\partial r}	+ \dfrac{\partial \hat{r}}{\partial \theta}\dfrac{1}{r}\dfrac{\partial u_{x}}{\partial r} \right)\hat{x} + \hat{\theta}\cdot\hat{r}\dfrac{\partial u_{x}}{\partial r}\dfrac{1}{r}\dfrac{\partial \hat{x}}{\partial \theta} \\
		%
		&+ \hat{\theta}\cdot\left(\hat{\theta}\dfrac{1}{r}\dfrac{\partial }{\partial \theta}\left(\dfrac{1}{r}\dfrac{\partial u_{x}}{\partial \theta}\right)\hat{x}
		+ \dfrac{\partial \hat{\theta}}{\partial \theta}\dfrac{1}{r^{2}}\dfrac{\partial u_{x}}{\partial \theta} \right) \hat{x} + \hat{\theta}\cdot\hat{\theta}\dfrac{1}{r}\dfrac{\partial u_{x}}{\partial \theta}\dfrac{1}{r}\dfrac{\partial \hat{x}}{\partial \theta}
	\end{split} \nonumber \\
	= \left[\dfrac{\partial^{2}u_{x}}{\partial x^{2}} + \dfrac{\partial^{2}u_{x}}{\partial r^{2}} + \dfrac{1}{r}\dfrac{\partial u_{x}}{\partial r} + \dfrac{1}{r^{2}}\dfrac{\partial^{2} u_{x}}{\partial \theta^{2}}	\right]\hat{x}
\end{eqnarray}




Second column:
\begin{eqnarray}
		\left(\hat{x}\cdot\dfrac{\partial}{\partial x} + \hat{r}\cdot\dfrac{\partial}{\partial r} + \hat{\theta}\cdot\dfrac{1}{r}\dfrac{\partial}{\partial \theta}\right)
		\left(	\hat{x}\dfrac{\partial u_{r}}{\partial x}  + \hat{r}\dfrac{\partial u_{r}}{\partial r} - \hat{\theta} \dfrac{u_{\theta}}{r} + \hat{\theta}\dfrac{1}{r}\dfrac{\partial u_{r}}{\partial \theta}
\right) \hat{r} \nonumber \\
%
	\begin{split}
	= & \quad \hat{x}\cdot\left(\hat{x}\dfrac{\partial^{2}u_{r}}{\partial x^{2}} + \dfrac{\partial \hat{x}}{\partial x}\dfrac{\partial u_{r}}{\partial x} \right)\hat{r} + \hat{x}\cdot\hat{x}\dfrac{\partial u_{r}}{\partial x}\dfrac{\partial \hat{r}}{\partial x} \\
	%
	&+ \hat{x}\cdot\left(\hat{r}\dfrac{\partial }{\partial x}\dfrac{\partial u_{r}}{\partial r} + \dfrac{\partial \hat{r}}{\partial x}\dfrac{\partial u_{r}}{\partial r}\right)\hat{r} + \hat{x}\cdot\hat{r}\dfrac{\partial u_{r}}{\partial r}\dfrac{\partial \hat{r}}{\partial x} \\
	%
	& + \hat{x}\cdot\left( -\hat{\theta}\dfrac{\partial }{\partial x}\dfrac{u_{\theta}}{r}	- \dfrac{\partial \hat{\theta}}{\partial x}\dfrac{u_{\theta}}{r}  \right)\hat{r}   - \hat{x}\cdot\hat{\theta}\dfrac{u_{\theta}}{r}\dfrac{\partial \hat{r}}{\partial x} \\
	%
	&+ \hat{x}\cdot\left(\hat{\theta}\dfrac{\partial }{\partial x}\left(\frac{1}{r}\dfrac{\partial u_{r}}{\partial \theta}\right) + \dfrac{\partial \hat{\theta}}{\partial x}\dfrac{1}{r}\dfrac{\partial u_{r}}{\partial \theta} \right)\hat{r}  + \hat{x}\cdot\hat{\theta}\dfrac{1}{r}\dfrac{\partial u_{r}}{\partial \theta}\dfrac{\partial \hat{r}}{\partial x}\\
	%
	&+ \hat{r}\cdot\left(\hat{x}\dfrac{\partial }{\partial r}\dfrac{\partial u_{r}}{\partial x}	+ \dfrac{\partial \hat{x}}{\partial r}\dfrac{\partial u_{r}}{\partial x}	\right)\hat{r} + \hat{r}\cdot\hat{x}\dfrac{\partial u_{r}}{\partial x}\dfrac{\partial \hat{r}}{\partial r} \\
	%
	&+ \hat{r}\cdot\left(\hat{r}\dfrac{\partial^{2}u_{r}}{\partial r^{2}} + \dfrac{\partial \hat{r}}{\partial r}\dfrac{\partial u_{r}}{\partial r}\right)\hat{r} + \hat{r}\cdot\hat{r}\dfrac{\partial u_{r}}{\partial r}\dfrac{\partial \hat{r}}{\partial r}\\
	%
	& + \hat{r}\cdot\left( -\hat{\theta}\dfrac{\partial }{\partial r}\dfrac{u_{\theta}}{r}	- \dfrac{\partial \hat{\theta}}{\partial r}\dfrac{u_{\theta}}{r} \right)\hat{r}  - \hat{r}\cdot\hat{\theta}\dfrac{u_{\theta}}{r}\dfrac{\partial \hat{r}}{\partial r}  \\
	%
	&+ \hat{r}\cdot\left(\hat{\theta}\dfrac{\partial }{\partial r}\left(\frac{1}{r}\dfrac{\partial u_{r}}{\partial \theta}\right) + \dfrac{\partial \hat{\theta}}{\partial r}\dfrac{1}{r}\dfrac{\partial u_{r}}{\partial \theta} \right)\hat{r} + \hat{r}\cdot\hat{\theta}\dfrac{1}{r}\dfrac{\partial u_{r}}{\partial \theta}\dfrac{\partial \hat{r}}{\partial r}\\
	%
	&+ \hat{\theta}\cdot\left(\hat{x}\dfrac{1}{r}\dfrac{\partial }{\partial \theta}\dfrac{\partial u_{r}}{\partial x}  + \dfrac{\partial \hat{x}}{\partial \theta}\dfrac{1}{r}\dfrac{\partial u_{r}}{\partial x} \right)\hat{r} + \hat{\theta}\cdot\hat{x}\dfrac{\partial u_{r}}{\partial x}\dfrac{1}{r}\dfrac{\partial \hat{r}}{\partial \theta} \\
	%
	&+ \hat{\theta}\cdot\left(	\hat{r}\dfrac{1}{r}\dfrac{\partial}{\partial \theta}\dfrac{\partial u_{r}}{\partial r}	+ \dfrac{\partial \hat{r}}{\partial \theta}\dfrac{1}{r}\dfrac{\partial u_{r}}{\partial r} \right)\hat{r} + \hat{\theta}\cdot\hat{r}\dfrac{\partial u_{r}}{\partial r}\dfrac{1}{r}\dfrac{\partial \hat{r}}{\partial \theta} \\
	%
	& + \hat{\theta}\cdot\left( -\hat{\theta}\dfrac{1}{r}\dfrac{\partial }{\partial \theta}\dfrac{u_{\theta}}{r}	- \dfrac{\partial \hat{\theta}}{\partial \theta}\dfrac{1}{r}\dfrac{u_{\theta}}{r}  \right)\hat{r}  - \hat{\theta}\cdot\hat{\theta}\dfrac{u_{\theta}}{r}\dfrac{1}{r}\dfrac{\partial \hat{r}}{\partial \theta} \\
	%
	&+ \hat{\theta}\cdot\left(\hat{\theta}\dfrac{1}{r}\dfrac{\partial }{\partial \theta}\left(\dfrac{1}{r}\dfrac{\partial u_{r}}{\partial \theta}\right)
	+ \dfrac{\partial \hat{\theta}}{\partial \theta}\dfrac{1}{r^{2}}\dfrac{\partial u_{r}}{\partial \theta} \right) \hat{r} + \hat{\theta}\cdot\hat{\theta}\dfrac{1}{r}\dfrac{\partial u_{r}}{\partial \theta}\dfrac{1}{r}\dfrac{\partial \hat{r}}{\partial \theta}
\end{split} \nonumber \\
%
= \left[\dfrac{\partial^{2}u_{r}}{\partial x^{2}} + \dfrac{\partial^{2}u_{r}}{\partial r^{2}} + \dfrac{1}{r}\dfrac{\partial u_{r}}{\partial r} - \dfrac{1}{r^{2}}\dfrac{\partial u_{\theta}}{\partial \theta} +\dfrac{1}{r^{2}}\dfrac{\partial^{2} u_{r}}{\partial \theta^{2}} 	\right]\hat{r}  + \left[- \dfrac{u_{\theta}}{r^{2}} + \dfrac{1}{r^{2}}\dfrac{\partial u_{r}}{\partial \theta}\right]\hat{\theta}
\end{eqnarray}



Third column:
\begin{eqnarray}
		\left(\hat{x}\cdot\dfrac{\partial}{\partial x} + \hat{r}\cdot\dfrac{\partial}{\partial r} + \hat{\theta}\cdot\dfrac{1}{r}\dfrac{\partial}{\partial \theta}\right)
\left(	\hat{x}\dfrac{\partial u_{\theta}}{\partial x}  + \hat{r}\dfrac{\partial u_{\theta}}{\partial r} + \hat{\theta} \dfrac{u_{r}}{r} + \hat{\theta}\dfrac{1}{r}\dfrac{\partial u_{\theta}}{\partial \theta}
\right) \hat{\theta} \nonumber \\
%
\begin{split}
	= & \quad \hat{x}\cdot\left(\hat{x}\dfrac{\partial^{2}u_{\theta}}{\partial x^{2}} + \dfrac{\partial \hat{x}}{\partial x}\dfrac{\partial u_{\theta}}{\partial x} \right)\hat{\theta} + \hat{x}\cdot\hat{x}\dfrac{\partial u_{\theta}}{\partial x}\dfrac{\partial \hat{\theta}}{\partial x} \\
	%
	&+ \hat{x}\cdot\left(\hat{r}\dfrac{\partial }{\partial x}\dfrac{\partial u_{\theta}}{\partial r} + \dfrac{\partial \hat{r}}{\partial x}\dfrac{\partial u_{\theta}}{\partial r}\right)\hat{\theta} + \hat{x}\cdot\hat{r}\dfrac{\partial u_{\theta}}{\partial r}\dfrac{\partial \hat{\theta}}{\partial x} \\
	%
	& + \hat{x}\cdot\left( \hat{\theta}\dfrac{\partial }{\partial x}\dfrac{u_{r}}{r}	+ \dfrac{\partial \hat{\theta}}{\partial x}\dfrac{u_{r}}{r}  \right)\hat{\theta}   + \hat{x}\cdot\hat{\theta}\dfrac{u_{r}}{r}\dfrac{\partial \hat{\theta}}{\partial x} \\
	%
	&+ \hat{x}\cdot\left(\hat{\theta}\dfrac{\partial }{\partial x}\left(\frac{1}{r}\dfrac{\partial u_{\theta}}{\partial \theta}\right) + \dfrac{\partial \hat{\theta}}{\partial x}\dfrac{1}{r}\dfrac{\partial u_{\theta}}{\partial \theta} \right)\hat{\theta} + \hat{x}\cdot\hat{\theta}\dfrac{1}{r}\dfrac{\partial u_{\theta}}{\partial \theta}\dfrac{\partial \hat{\theta}}{\partial x}\\
	%
	&+ \hat{r}\cdot\left(\hat{x}\dfrac{\partial }{\partial r}\dfrac{\partial u_{\theta}}{\partial x} + \dfrac{\partial \hat{x}}{\partial r}\dfrac{\partial u_{\theta}}{\partial x}	\right)\hat{\theta} + \hat{r}\cdot\hat{x}\dfrac{\partial u_{\theta}}{\partial x}\dfrac{\partial \hat{\theta}}{\partial r} \\
	%
	&+ \hat{r}\cdot\left(\hat{r}\dfrac{\partial^{2}u_{\theta}}{\partial r^{2}} + \dfrac{\partial \hat{r}}{\partial r}\dfrac{\partial u_{\theta}}{\partial r}\right)\hat{\theta} + \hat{r}\cdot\hat{r}\dfrac{\partial u_{\theta}}{\partial r}\dfrac{\partial \hat{\theta}}{\partial r}\\
	%
	& + \hat{r}\cdot\left( \hat{\theta}\dfrac{\partial }{\partial r}\dfrac{u_{r}}{r}  + \dfrac{\partial \hat{\theta}}{\partial r}\dfrac{u_{r}}{r} \right)\hat{\theta}  + \hat{r}\cdot\hat{\theta}\dfrac{u_{r}}{r}\dfrac{\partial \hat{\theta}}{\partial r}  \\
	%
	&+ \hat{r}\cdot\left(\hat{\theta}\dfrac{\partial }{\partial r}\left(\frac{1}{r}\dfrac{\partial u_{\theta}}{\partial \theta}\right) + \dfrac{\partial \hat{\theta}}{\partial r}\dfrac{1}{r}\dfrac{\partial u_{\theta}}{\partial \theta} \right)\hat{\theta} + \hat{r}\cdot\hat{\theta}\dfrac{1}{r}\dfrac{\partial u_{\theta}}{\partial \theta}\dfrac{\partial \hat{\theta}}{\partial r}\\
	%
	&+ \hat{\theta}\cdot\left(\hat{x}\dfrac{1}{r}\dfrac{\partial }{\partial \theta}\dfrac{\partial u_{\theta}}{\partial x}  + \dfrac{\partial \hat{x}}{\partial \theta}\dfrac{1}{r}\dfrac{\partial u_{\theta}}{\partial x} \right)\hat{\theta} + \hat{\theta}\cdot\hat{x}\dfrac{\partial u_{\theta}}{\partial x}\dfrac{1}{r}\dfrac{\partial \hat{\theta}}{\partial \theta} \\
	%
	&+ \hat{\theta}\cdot\left(	\hat{r}\dfrac{1}{r}\dfrac{\partial}{\partial \theta}\dfrac{\partial u_{\theta}}{\partial r}	+ \dfrac{\partial \hat{r}}{\partial \theta}\dfrac{1}{r}\dfrac{\partial u_{\theta}}{\partial r} \right)\hat{\theta} + \hat{\theta}\cdot\hat{r}\dfrac{\partial u_{\theta}}{\partial r}\dfrac{1}{r}\dfrac{\partial \hat{\theta}}{\partial \theta} \\
	%
	& + \hat{\theta}\cdot\left( \hat{\theta}\dfrac{1}{r}\dfrac{\partial }{\partial \theta}\dfrac{u_{r}}{r}	+ \dfrac{\partial \hat{\theta}}{\partial \theta}\dfrac{1}{r}\dfrac{u_{r}}{r}  \right)\hat{\theta}  + \hat{\theta}\cdot\hat{\theta}\dfrac{u_{r}}{r}\dfrac{1}{r}\dfrac{\partial \hat{\theta}}{\partial \theta} \\
	%
	&+ \hat{\theta}\cdot\left(\hat{\theta}\dfrac{1}{r}\dfrac{\partial }{\partial \theta}\left(\dfrac{1}{r}\dfrac{\partial u_{\theta}}{\partial \theta}\right)
	+ \dfrac{\partial \hat{\theta}}{\partial \theta}\dfrac{1}{r^{2}}\dfrac{\partial u_{\theta}}{\partial \theta} \right) \hat{\theta} + \hat{\theta}\cdot\hat{\theta}\dfrac{1}{r}\dfrac{\partial u_{\theta}}{\partial \theta}\dfrac{1}{r}\dfrac{\partial \hat{\theta}}{\partial \theta}
\end{split} \nonumber \\
	%
	= \left[\dfrac{\partial^{2}u_{\theta}}{\partial x^{2}} + \dfrac{\partial^{2}u_{\theta}}{\partial r^{2}} + \dfrac{1}{r}\dfrac{\partial u_{\theta}}{\partial r} + \dfrac{1}{r^{2}}\dfrac{\partial u_{r}}{\partial \theta}   + \dfrac{1}{r^{2}}\dfrac{\partial^{2} u_{\theta}}{\partial \theta^{2}}	\right]\hat{\theta}  + \left[- \dfrac{u_{r}}{r^{2}}  - \dfrac{1}{r^{2}}\dfrac{\partial u_{\theta}}{\partial \theta}  \right]\hat{r}
\end{eqnarray}

% Collected vector laplacian quantities.
\begin{eqnarray}
	\therefore \nabla\cdot\nabla (\bm{u}) = \left\{
	\begin{split}
		& \left[\dfrac{\partial^{2}u_{x}}{\partial x^{2}} + \dfrac{1}{r}\dfrac{\partial }{\partial r}\left(r\dfrac{\partial u_{x}}{\partial r}\right) +  \dfrac{1}{r^{2}}\dfrac{\partial^{2} u_{x}}{\partial \theta^{2}}	\right]\hat{x} + \\
		%
		& \left[\dfrac{\partial^{2}u_{r}}{\partial x^{2}} + \dfrac{1}{r}\dfrac{\partial }{\partial r}\left(r\dfrac{\partial u_{r}}{\partial r}\right) +\dfrac{1}{r^{2}}\dfrac{\partial^{2} u_{r}}{\partial \theta^{2}} - \dfrac{u_{r}}{r^{2}} - \dfrac{2}{r^{2}}\dfrac{\partial u_{\theta}}{\partial \theta}  \right]\hat{r}  +\\
		%
		& \left[\dfrac{\partial^{2}u_{\theta}}{\partial x^{2}} + \dfrac{1}{r}\dfrac{\partial }{\partial r}\left(r\dfrac{\partial u_{\theta}}{\partial r}\right) + \dfrac{1}{r^{2}}\dfrac{\partial^{2} u_{\theta}}{\partial \theta^{2}} + \dfrac{2}{r^{2}}\dfrac{\partial u_{r}}{\partial \theta}   - \dfrac{u_{\theta}}{r^{2}} \right]\hat{\theta}  
	\end{split}\right. \label{eqn:vector_laplacian}
\end{eqnarray}

Which can be written using the Laplacian of individual scalar fields, $u_{x},u_{r},u_{\theta}$ as,

\begin{eqnarray}
	\therefore \nabla\cdot\nabla (\bm{u}) = 
\begin{split}
	 \nabla^{2}(u_{x}) \hat{x} + \left(\nabla^{2}(u_{r}) - \dfrac{u_{r}}{r^{2}} - \dfrac{2}{r^{2}}\dfrac{\partial u_{\theta}}{\partial \theta}  \right)\hat{r}  + 
	%
	 \left( \nabla^{2}(u_{\theta}) + \dfrac{2}{r^{2}}\dfrac{\partial u_{r}}{\partial \theta}   - \dfrac{u_{\theta}}{r^{2}} \right)\hat{\theta}  
\end{split} \label{eqn:vector_laplacian_scalar}
\end{eqnarray}

For the weak Laplacian we need the gradient of the test functions:
\begin{subequations}
\begin{eqnarray}
	\nabla(v_{x}\hat{x})\bigcdot =& \left(\hat{x}\dfrac{\partial v_{x}}{\partial x} \hat{x}\bigcdot + \hat{x}\dfrac{\partial v_{x}}{\partial r}\hat{r}\bigcdot + \hat{x}\dfrac{1}{r}\dfrac{\partial v_{x}}{\partial \theta}\hat{\theta}\bigcdot \right)  \\
	%
	\nabla(v_{r}\hat{r})\bigcdot = &\left(\hat{r}\dfrac{\partial v_{r}}{\partial x} \hat{x}\bigcdot + \hat{r}\dfrac{\partial v_{r}}{\partial r}\hat{r}\bigcdot + \hat{r}\dfrac{1}{r}\dfrac{\partial v_{r}}{\partial \theta}\hat{\theta}\bigcdot  + \hat{\theta}\dfrac{v_{r}}{r}\hat{\theta}\bigcdot \right) \\
	%
	\nabla(v_{\theta}\hat{\theta})\bigcdot =&
	 \left(\hat{\theta}\dfrac{\partial v_{\theta}}{\partial x} \hat{x}\bigcdot + \hat{\theta}\dfrac{\partial v_{\theta}}{\partial r}\hat{r}\bigcdot + \hat{\theta}\dfrac{1}{r}\dfrac{\partial v_{\theta}}{\partial \theta}\hat{\theta}\bigcdot  - \hat{r}\dfrac{v_{\theta}}{r}\hat{\theta}\bigcdot  \right)
	%
	\end{eqnarray}
\end{subequations}

Strain rate tensor $\mathcal{S} = 1/2(\nabla \bm{u} + \nabla \bm{u}^{T})$:

\begin{eqnarray}
	2\mathcal{S} = \left\{
	\begin{split}
		& \left(\hat{x}\dfrac{\partial u_{x}}{\partial x}\hat{x} + \hat{r}\dfrac{\partial u_{x}}{\partial r}\hat{x} + \hat{\theta}\dfrac{1}{r}\dfrac{\partial u_{x}}{\partial \theta}\hat{x}	\right) 
		%
		+ \left(\hat{x}\dfrac{\partial u_{x}}{\partial x}\hat{x} + \hat{x}\dfrac{\partial u_{x}}{\partial r}\hat{r} + \hat{x}\dfrac{1}{r}\dfrac{\partial u_{x}}{\partial \theta}\hat{\theta}	\right) \\
		& \left(\hat{x}\dfrac{\partial u_{r}}{\partial x}\hat{r} + \hat{r}\dfrac{\partial u_{r}}{\partial r}\hat{r} + \hat{\theta}\dfrac{1}{r}\dfrac{\partial u_{r}}{\partial \theta} \hat{r}	 + \hat{\theta}\dfrac{u_{r}}{r}\hat{\theta}	\right) 
		%
		+ \left(\hat{r}\dfrac{\partial u_{r}}{\partial x}\hat{x} + \hat{r}\dfrac{\partial u_{r}}{\partial r}\hat{r} + \hat{r}\dfrac{1}{r}\dfrac{\partial u_{r}}{\partial \theta} \hat{\theta} + \hat{\theta}\dfrac{u_{r}}{r}\hat{\theta}	\right) \\
		& \left(\hat{x}\dfrac{\partial u_{\theta}}{\partial x}\hat{\theta} + \hat{r}\dfrac{\partial u_{\theta}}{\partial r}\hat{\theta} + \hat{\theta}\dfrac{1}{r}\dfrac{\partial u_{\theta}}{\partial \theta}\hat{\theta} - \hat{\theta}\dfrac{u_{\theta}}{r}\hat{r}	\right)
		%
		+ \left(\hat{\theta}\dfrac{\partial u_{\theta}}{\partial x}\hat{x} + \hat{\theta}\dfrac{\partial u_{\theta}}{\partial r}\hat{r} + \hat{\theta}\dfrac{1}{r}\dfrac{\partial u_{\theta}}{\partial \theta}\hat{\theta} - \hat{r}\dfrac{u_{\theta}}{r}\hat{\theta}	\right)
	\end{split}\right.
\end{eqnarray}

Rearranging terms for convenience:
%\begin{eqnarray}
%	2\mathcal{S} = \left\{
%	\begin{split}
%		& \hat{x}\left(2\dfrac{\partial u_{x}}{\partial x}\right)\hat{x} + \\
%		& \hat{x}\left(\dfrac{\partial u_{x}}{\partial r} + \dfrac{\partial u_{r}}{\partial x}			\right)\hat{r} + \\
%		& \hat{x}\left(\dfrac{\partial u_{\theta}}{\partial x} + \dfrac{1}{r}\dfrac{\partial u_{x}}{\partial \theta} \right)\hat{\theta} + \\
%		& \hat{r}\left(\dfrac{\partial u_{x}}{\partial r} + \dfrac{\partial u_{r}}{\partial x}			\right)\hat{x} + \\
%		& \hat{r}\left(2\dfrac{\partial u_{r}}{\partial r}\right) \hat{r} + \\
%		& \hat{r}\left(\dfrac{\partial u_{\theta}}{\partial r} + \dfrac{1}{r}\dfrac{\partial u_{r}}{\partial \theta} - \dfrac{u_{\theta}}{r}	\right)\hat{\theta} + \\
%		& \hat{\theta}\left(\dfrac{1}{r}\dfrac{\partial u_{x}}{\partial \theta} + \dfrac{\partial u_{\theta}}{\partial x} \right)\hat{x} + \\
%		& \hat{\theta}\left(\dfrac{1}{r}\dfrac{\partial u_{r}}{\partial \theta} + \dfrac{\partial u_{\theta}}{\partial r} - \dfrac{u_{\theta}}{r} \right) \hat{r} \\
%		& \hat{\theta}\left(\dfrac{2}{r}\dfrac{\partial u_{\theta}}{\partial \theta} + 2\dfrac{u_{r}}{r}	\right) \theta
%	\end{split}\right.
%\end{eqnarray}

\begin{eqnarray}
	2\mathcal{S} = \left\{
	\begin{split}
		% x
		& \quad \hat{x}\left(2\dfrac{\partial u_{x}}{\partial x}\right)\hat{x} &+& \hat{x}\left(\dfrac{\partial u_{x}}{\partial r} + \dfrac{\partial u_{r}}{\partial x}			\right)\hat{r}  &+& \hat{x}\left(\dfrac{\partial u_{\theta}}{\partial x} + \dfrac{1}{r}\dfrac{\partial u_{x}}{\partial \theta} \right)\hat{\theta}  \\	
		% R	
		& + \hat{r}\left(\dfrac{\partial u_{x}}{\partial r} + \dfrac{\partial u_{r}}{\partial x}			\right)\hat{x}  &+& \hat{r}\left(2\dfrac{\partial u_{r}}{\partial r}\right) \hat{r} & +& \hat{r}\left(\dfrac{\partial u_{\theta}}{\partial r} + \dfrac{1}{r}\dfrac{\partial u_{r}}{\partial \theta} - \dfrac{u_{\theta}}{r}	\right)\hat{\theta} \\
		% \theta
		& + \hat{\theta}\left(\dfrac{1}{r}\dfrac{\partial u_{x}}{\partial \theta} + \dfrac{\partial u_{\theta}}{\partial x} \right)\hat{x} & + & \hat{\theta}\left(\dfrac{1}{r}\dfrac{\partial u_{r}}{\partial \theta} + \dfrac{\partial u_{\theta}}{\partial r} - \dfrac{u_{\theta}}{r} \right) \hat{r} &+& \hat{\theta}\left(\dfrac{2}{r}\dfrac{\partial u_{\theta}}{\partial \theta} + 2\dfrac{u_{r}}{r}	\right)\hat{\theta}
	\end{split}\right.
\end{eqnarray}

Now taking the dot product with the gradient of the test function, $\nabla (v)\bigcdot$
\begin{subequations}
\begin{eqnarray}
	\nabla(v_{x}\hat{x})\bigcdot\mathcal{S} = & \left\{
	\begin{split}
	\dfrac{\partial v_{x}}{\partial x}\dfrac{\partial u_{x}}{\partial x} 
	+ \dfrac{1}{2}\dfrac{\partial v_{x}}{\partial r}\left(\dfrac{\partial u_{x}}{\partial r} + \dfrac{\partial u_{r}}{\partial x}	\right) \\
	+ \dfrac{1}{2r}\dfrac{\partial v_{x}}{\partial \theta}\left(\dfrac{1}{r}\dfrac{\partial u_{x}}{\partial \theta} + \dfrac{\partial u_{\theta}}{\partial x} \right) 
		\end{split} \right. \\
	%
	\nabla(v_{r}\hat{r})\bigcdot\mathcal{S} = & \left\{
	\begin{split}
	\dfrac{1}{2}\dfrac{\partial v_{r}}{\partial x}\left(\dfrac{\partial u_{x}}{\partial r} + \dfrac{\partial u_{r}}{\partial x} \right)
	+ \dfrac{\partial v_{r}}{\partial r}\dfrac{\partial u_{r}}{\partial r} \\
	+ \dfrac{1}{2r}\dfrac{\partial v_{r}}{\partial \theta}\left(\dfrac{1}{r}\dfrac{\partial u_{r}}{\partial \theta} + \dfrac{\partial u_{\theta}}{\partial r} - \dfrac{u_{\theta}}{r} \right) 
	+ \dfrac{v_{r}}{r}\left(\dfrac{1}{r}\dfrac{\partial u_{\theta}}{\partial \theta} + \dfrac{u_{r}}{r}	\right)
		\end{split}\right. \\
	%
	\nabla(v_{\theta}\hat{\theta})\bigcdot\mathcal{S} =&  \left\{
	\begin{split}
	\dfrac{1}{2}\dfrac{\partial v_{\theta}}{\partial x}\left(\dfrac{\partial u_{\theta}}{\partial x} + \dfrac{1}{r}\dfrac{\partial u_{x}}{\partial \theta} \right) 
	+ \dfrac{1}{2}\dfrac{\partial v_{\theta}}{\partial r}\left(\dfrac{\partial u_{\theta}}{\partial r} + \dfrac{1}{r}\dfrac{\partial u_{r}}{\partial \theta} - \dfrac{u_{\theta}}{r}	\right) \\
	+ \dfrac{1}{r}\dfrac{\partial v_{\theta}}{\partial \theta}\left(\dfrac{1}{r}\dfrac{\partial u_{\theta}}{\partial \theta} + \dfrac{u_{r}}{r}	\right)
	- \dfrac{v_{\theta}}{2r} \left(\dfrac{1}{r}\dfrac{\partial u_{r}}{\partial \theta} + \dfrac{\partial u_{\theta}}{\partial r} - \dfrac{u_{\theta}}{r} \right) 
	\end{split}\right.
\end{eqnarray}
\end{subequations}

%
%\begin{eqnarray}
%	\nabla(v_{x}\hat{x})\cdot\nabla (u_{x}\hat{x}) =& \left(\hat{x}\dfrac{\partial v_{x}}{\partial x} \hat{x} + \hat{r}\dfrac{\partial v_{x}}{\partial r}\hat{x} + \hat{\theta}\dfrac{1}{r}\dfrac{\partial v_{x}}{\partial \theta}\hat{x} \right) \cdot
%	%
%	 \left(\hat{x}\dfrac{\partial u_{x}}{\partial x} \hat{x} + \hat{r}\dfrac{\partial u_{x}}{\partial r}\hat{x} + \hat{\theta}\dfrac{1}{r}\dfrac{\partial u_{x}}{\partial \theta}\hat{x} \right) \nonumber \\
%	 =& \hat{x}\dfrac{\partial v_{x}}{\partial x}\dfrac{\partial u_{x}}{\partial x}\hat{x} + \hat{x}\dfrac{\partial v_{x}}{\partial r}\dfrac{\partial u_{x}}{\partial r}\hat{x} + \hat{x}\dfrac{1}{r^{2}}\dfrac{\partial v_{x}}{\partial \theta}\dfrac{\partial u_{x}}{\partial \theta}\hat{x} 
%\end{eqnarray}
%
%\begin{eqnarray}
%	\nabla(v_{r}\hat{r})\cdot\nabla (u_{r}\hat{r}) = &\left(\hat{x}\dfrac{\partial v_{r}}{\partial x} \hat{r} + \hat{r}\dfrac{\partial v_{r}}{\partial r}\hat{r} + \hat{\theta}\dfrac{1}{r}\dfrac{\partial v_{r}}{\partial \theta}\hat{r}  + \hat{\theta}\dfrac{v_{r}}{r}\hat{\theta} \right) \cdot
%	%
%	\left(\hat{x}\dfrac{\partial u_{r}}{\partial x} \hat{r} + \hat{r}\dfrac{\partial u_{r}}{\partial r}\hat{r} + \hat{\theta}\dfrac{1}{r}\dfrac{\partial u_{r}}{\partial \theta}\hat{r}  + \hat{\theta}\dfrac{u_{r}}{r}\hat{\theta} \right) \nonumber \\
%	= &\left\{
%	\begin{split}
%	 \hat{r}\dfrac{\partial v_{r}}{\partial x}\dfrac{\partial u_{r}}{\partial x}\hat{r} + \hat{r}\dfrac{\partial v_{r}}{\partial r}\dfrac{\partial u_{r}}{\partial r}\hat{r} + \hat{r}\dfrac{1}{r^{2}}\dfrac{\partial v_{r}}{\partial \theta}\dfrac{\partial u_{r}}{\partial \theta}\hat{r} + \\
%	\hat{\theta}\dfrac{v_{r}}{r}\dfrac{u_{r}}{r}\hat{\theta} 
%	+\hat{r}\dfrac{1}{r}\dfrac{\partial v_{r}}{\partial \theta}\dfrac{u_{r}}{r}\hat{\theta}
%	+ \hat{\theta}\dfrac{v_{r}}{r}\dfrac{1}{r}\dfrac{\partial u_{r}}{\partial \theta}\hat{r} 	\end{split}\right.
%\end{eqnarray}
%
%\begin{eqnarray}
%	\nabla(v_{\theta}\hat{\theta})\cdot\nabla (u_{\theta}\hat{\theta}) =&
%	%
%	 \left(\hat{x}\dfrac{\partial v_{\theta}}{\partial x} \hat{\theta} + \hat{r}\dfrac{\partial v_{\theta}}{\partial r}\hat{\theta} + \hat{\theta}\dfrac{1}{r}\dfrac{\partial v_{\theta}}{\partial \theta}\hat{\theta}  - \hat{\theta}\dfrac{v_{\theta}}{r}\hat{r}  \right)\cdot
%	\left(\hat{x}\dfrac{\partial u_{\theta}}{\partial x} \hat{\theta} + \hat{r}\dfrac{\partial u_{\theta}}{\partial r}\hat{\theta} + \hat{\theta}\dfrac{1}{r}\dfrac{\partial u_{\theta}}{\partial \theta}\hat{\theta}  - \hat{\theta}\dfrac{u_{\theta}}{r}\hat{r} \right) \nonumber \\
%	%
%	= &\left\{
%	\begin{split}
%			 \hat{\theta}\dfrac{\partial v_{\theta}}{\partial x}\dfrac{\partial u_{\theta}}{\partial x}\hat{\theta} + \hat{\theta}\dfrac{\partial v_{\theta}}{\partial r}\dfrac{\partial u_{\theta}}{\partial r}\hat{\theta} + \hat{\theta}\dfrac{1}{r^{2}}\dfrac{\partial v_{\theta}}{\partial \theta}\dfrac{\partial u_{\theta}}{\partial \theta}\hat{\theta} \\
%			 %
%			 + \hat{r}\dfrac{v_{\theta}}{r}\dfrac{u_{\theta}}{r}\hat{r}  -\hat{\theta}\dfrac{1}{r}\dfrac{\partial v_{\theta}}{\partial \theta}\dfrac{u_{\theta}}{r}\hat{r} 
%			 -\hat{r}\dfrac{v_{\theta}}{r}\dfrac{1}{r}\dfrac{\partial u_{\theta}}{\partial \theta}\hat{\theta}
%	\end{split}\right.
%\end{eqnarray}


\section{Nek Implementations:}

\subsection{Opgradt}

The opgradt subroutine implements the gradient of a field defined on Mesh 2 (pressure mesh), integrated w.r.t the Mesh 1 test functions.
This is done with integration by parts which creates a boundary term.

Pressure term in the momentum equations:
\begin{eqnarray}
	 - \mathlarger\int_{\Omega} v\dfrac{\partial }{\partial x_{j}}(p\delta_{ij}) d\Omega \nonumber \\
				  -\mathlarger\int_{\Omega} \dfrac{\partial }{\partial x_{j}}(vp\delta_{ij}) d\Omega + \mathlarger\int_{\Omega} p\delta_{ij}\dfrac{\partial }{\partial x_{j}}(v) d\Omega \nonumber \\
				  -\mathlarger\int_{\partial\Omega} (vp\delta_{ij}) n_{j} dA + \mathlarger\int_{\Omega} p\delta_{ij}\dfrac{\partial }{\partial x_{j}}(v) d\Omega \nonumber 
\end{eqnarray}
The first term is the boundary condition. The second term is what opgradt evaluates. I represent the determinant of the mapping between reference element and physical coordinates, the Jacobian as $\mathcal{J}$. In the continuous formulation, this becomes:
\begin{eqnarray}
	w_{i}	=& \mathlarger\int_{\Omega} p\dfrac{\partial v}{\partial x_{i}} d\Omega \nonumber \\
				 =& \mathlarger\int_{\Omega} p\dfrac{\partial v}{\partial x_{i}} \left(\dfrac{\partial \Omega}{\partial \hat{\Omega}}\right) d\hat{\Omega} \nonumber \\
				 =& \mathlarger\int_{\Omega} p\left(\dfrac{1}{\mathcal{J}}\dfrac{\partial r}{\partial x_{i}}\dfrac{\partial v}{\partial r} + \dfrac{1}{\mathcal{J}}\dfrac{\partial s}{\partial x_{i}}\dfrac{\partial v}{\partial s} + \dfrac{1}{\mathcal{J}}\dfrac{\partial t}{\partial x_{i}}\dfrac{\partial v}{\partial t}\right)\mathcal{J} d\hat{\Omega} \nonumber \\
				=& \mathlarger\int_{\Omega} p\left(\dfrac{\partial r}{\partial x_{i}}\dfrac{\partial v}{\partial r} + \dfrac{\partial s}{\partial x_{i}}\dfrac{\partial v}{\partial s} + \dfrac{\partial t}{\partial x_{i}}\dfrac{\partial v}{\partial t}\right) d\hat{\Omega} \nonumber \\
\implies w_{i}	=& \mathlarger\int_{\Omega} p\left(\dfrac{\partial r_{j}}{\partial x_{i}}\dfrac{\partial v}{\partial r_{j}}\right) d\hat{\Omega},
\end{eqnarray}
where, $r_{j}$ represents the reference element coordinate directions $r,s,t$.
After discretization, this becomes:
\begin{eqnarray}
	\implies w_{i}	=& \mathlarger\sum_{k} W(x_{k})\left(\dfrac{\partial r_{j}}{\partial x_{i}}\dfrac{\partial v}{\partial r_{j}}(x_{k})\right)p  \nonumber
\end{eqnarray}
When cross derivative terms like $\partial s/\partial x$ \textit{etc.} are non-zero, we have the following expression
\begin{eqnarray}
	\begin{split}
		w_{i}	=&  (\mathcal{I}^{T}_{t12}\otimes\mathcal{I}^{T}_{s12}\otimes\mathcal{D}^{T}_{r12})(\dfrac{\partial r}{\partial x_{i}}.*W.*p) + \\
		& (\mathcal{I}^{T}_{t12}\otimes\mathcal{D}^{T}_{s12}\otimes\mathcal{I}^{T}_{r12})(\dfrac{\partial s}{\partial x_{i}}.*W.*p) + \\
		& (\mathcal{D}^{T}_{t12}\otimes\mathcal{I}^{T}_{s12}\otimes\mathcal{I}^{T}_{r12})(\dfrac{\partial t}{\partial {x_{i}}}.*W.*p)			
	\end{split}
\end{eqnarray}

In the case of no cross derivative terms, the above expression simplifies to:
\begin{subequations}
	\begin{eqnarray}
		w_{1}	= (\mathcal{I}^{T}_{t12}\otimes\mathcal{I}^{T}_{s12}\otimes\mathcal{D}^{T}_{r12})(\dfrac{\partial r}{\partial x}.*W.*p) \\
		w_{2}	= (\mathcal{I}^{T}_{t12}\otimes\mathcal{D}^{T}_{s12}\otimes\mathcal{I}^{T}_{r12})(\dfrac{\partial s}{\partial y}.*W.*p) \\
		w_{3}	= (\mathcal{D}^{T}_{t12}\otimes\mathcal{I}^{T}_{s12}\otimes\mathcal{I}^{T}_{r12})(\dfrac{\partial t}{\partial z}.*W.*p)
	\end{eqnarray}
\end{subequations}
which can be further expressed purely as kronecker products, taking $\partial r/\partial x,\ \partial s/\partial y,\ \partial t/\partial z$ \textit{etc.} as diagonal matrices for the respective one dimensional problems.
\begin{subequations}
	\begin{eqnarray}
		w_{1}	= (\mathcal{I}^{T}_{t12}\mathcal{I}_{2}W_{t2}\otimes\mathcal{I}^{T}_{s12}\mathcal{I}_{2}W_{s2}\otimes\mathcal{D}^{T}_{r12}\dfrac{\partial r}{\partial x}W_{r2})p \\
		w_{2}	= (\mathcal{I}^{T}_{t12}\mathcal{I}_{2}W_{t2}\otimes\mathcal{D}^{T}_{s12}\dfrac{\partial s}{\partial y}W_{s2}\otimes\mathcal{I}^{T}_{r12}\mathcal{I}_{2}W_{r2})p \\
		w_{3}	= (\mathcal{D}^{T}_{t12}\dfrac{\partial t}{\partial z}W_{t2}\otimes\mathcal{I}^{T}_{s12}\mathcal{I}_{2}W_{s2}\otimes\mathcal{I}^{T}_{r12}\mathcal{I}_{2}W_{r2})p
	\end{eqnarray}
\end{subequations}

For the cylindrical case we encounter some differences. There is an additional factor of $R$ in front of the integral, \textit{i.e.} $\partial \Omega_{x,y,z} \rightarrow R\partial \Omega_{x,R,\theta} \rightarrow R\mathcal{J}\partial \hat{\Omega}$. Also, for the $\theta$ term we have an additional division by $R$. Finally, there is an additional term for $R$ component for the derivative of the unit vector.
\begin{eqnarray}
	w_{x} =& \mathlarger\int_{\Omega} p\left(\dfrac{\partial r}{\partial x}\dfrac{\partial v}{\partial r} + \dfrac{\partial s}{\partial x}\dfrac{\partial v}{\partial s} + \dfrac{\partial t}{\partial x}\dfrac{\partial v}{\partial t}\right) R d\hat{\Omega} \nonumber \\
	w_{R} =& \mathlarger\int_{\Omega} p\left(\dfrac{\partial r}{\partial R}\dfrac{\partial v}{\partial r} + \dfrac{\partial s}{\partial R}\dfrac{\partial v}{\partial s} + \dfrac{\partial t}{\partial R}\dfrac{\partial v}{\partial t} + \dfrac{v}{R} \right) R d\hat{\Omega} \nonumber \\
	w_{\theta} =& \mathlarger\int_{\Omega} p\left(\dfrac{\partial r}{\partial \theta}\dfrac{\partial v}{\partial r} + \dfrac{\partial s}{\partial \theta}\dfrac{\partial v}{\partial s} + \dfrac{\partial t}{\partial \theta}\dfrac{\partial v}{\partial t}\right) d\hat{\Omega} \nonumber
\end{eqnarray}

Which will lead to the similar (but not identical) expressions for opgradt. Substituting $\widetilde{\mathcal{D}}^{T}_{s12} = (\partial s/\partial y)\mathcal{D}^{T}_{s12} + \mathcal{I}^{T}_{s12}/R$, and if we assume the grid is Cartesian in the $x-R-\theta$ space, \textit{i.e.} there are no cross derivatives.
\begin{subequations}
	\begin{eqnarray}
		w_{x}	= (\mathcal{I}^{T}_{t12}\otimes\mathcal{I}^{T}_{s12}\otimes\mathcal{D}^{T}_{r12})(\dfrac{\partial r}{\partial x}.*R.*W.*p) \\
		w_{R}	= (\mathcal{I}^{T}_{t12}\otimes(\dfrac{\partial s}{\partial y}\mathcal{D}^{T}_{s12} + \dfrac{\mathcal{I}_{s12}}{R})\otimes\mathcal{I}^{T}_{r12})(R.*W.*p) \\
		w_{\theta}	= \mathcal{D}^{T}_{t12}\otimes\mathcal{I}^{T}_{s12}\otimes\mathcal{I}^{T}_{r12})(\dfrac{\partial t}{\partial z}.*W.*p)
	\end{eqnarray}
\end{subequations}
The factors of $R,W$ and the geometric factors can also be expressed in kronecker product form:
\begin{subequations}
	\begin{eqnarray}
		w_{x}	=& (\mathcal{I}^{T}_{t12}\otimes\mathcal{I}^{T}_{s12}\otimes\mathcal{D}^{T}_{r12})(\mathcal{I}_{2}\otimes\mathcal{I}_{2}\otimes\dfrac{\partial r}{\partial x})(\mathcal{I}_{2}\otimes R_{2}\otimes\mathcal{I}_{2})(W_{t2}\otimes W_{s2}\otimes W_{r2})p \\
					  =& (\mathcal{I}^{T}_{t12}\mathcal{I}_{2}\mathcal{I}_{2}\otimes\mathcal{I}^{T}_{s12}\mathcal{I}_{2}R_{2}\otimes\mathcal{D}^{T}_{r12}\dfrac{\partial r}{\partial x}\mathcal{I}_{2})(W_{t2}\otimes W_{s2}\otimes W_{r2})p \\
					  =& (\mathcal{I}^{T}_{t12}\mathcal{I}_{2}\mathcal{I}_{2}W_{t2}\otimes\mathcal{I}^{T}_{s12}\mathcal{I}_{2}R_{2}W_{s2}\otimes\mathcal{D}^{T}_{r12}\dfrac{\partial r}{\partial x}\mathcal{I}_{2}W_{r2})p \\
					  =& (\mathcal{I}^{T}_{t12}W_{t2}\otimes\mathcal{I}^{T}_{s12}R_{2}W_{s2}\otimes\mathcal{D}^{T}_{r12}\dfrac{\partial r}{\partial x}W_{r2})p					  
	\end{eqnarray}
\end{subequations}

$R$-direction:
\begin{subequations}
	\begin{eqnarray}
		w_{R}	=& (\mathcal{I}^{T}_{t12}\otimes\widetilde{\mathcal{D}}^{T}_{s12}\otimes\mathcal{I}^{T}_{r12})(\mathcal{I}_{2}\otimes\dfrac{\partial s}{\partial R}\otimes\mathcal{I}_{2})(\mathcal{I}_{2}\otimes R_{2}\otimes\mathcal{I}_{2})(W_{t2}\otimes W_{s2}\otimes W_{r2})p \\
	  	=& (\mathcal{I}^{T}_{t12}\mathcal{I}_{2}\otimes\widetilde{\mathcal{D}}^{T}_{s12}R_{2}\otimes\mathcal{I}^{T}_{r12}\mathcal{I}_{2})(W_{t2}\otimes W_{s2}\otimes W_{r2})p \\
		=& (\mathcal{I}^{T}_{t12}\mathcal{I}_{2}W_{t2}\otimes\widetilde{\mathcal{D}}^{T}_{s12}R_{2}W_{s2}\otimes\mathcal{I}^{T}_{r12}\mathcal{I}_{2}W_{r2})p \\
		=& (\mathcal{I}^{T}_{t12}W_{t2}\otimes\widetilde{\mathcal{D}}^{T}_{s12}R_{2}W_{s2}\otimes\mathcal{I}^{T}_{r12}W_{r2})p		
	\end{eqnarray}
\end{subequations}

$\theta$-direction:
\begin{subequations}
	\begin{eqnarray}
		w_{\theta}	=& (\mathcal{D}^{T}_{t12}\otimes\mathcal{I}^{T}_{s12}\otimes\mathcal{I}^{T}_{r12})(\dfrac{\partial t}{\partial \theta}\otimes\mathcal{I}_{2}\otimes\mathcal{I}_{2})(\mathcal{I}_{2}\otimes \mathcal{I}_{2}\otimes\mathcal{I}_{2})(W_{t2}\otimes W_{s2}\otimes W_{r2})p \\
		=& (\mathcal{D}^{T}_{t12}\dfrac{\partial t}{\partial \theta}\mathcal{I}_{2}\otimes\mathcal{I}^{T}_{s12}\mathcal{I}_{2}\mathcal{I}_{2}\otimes\mathcal{I}^{T}_{r12}\mathcal{I}_{2}\mathcal{I}_{2})(W_{t2}\otimes W_{s2}\otimes W_{r2})p \\
		=& (\mathcal{D}^{T}_{t12}\dfrac{\partial t}{\partial \theta}\mathcal{I}_{2}W_{t2}\otimes\mathcal{I}^{T}_{s12}\mathcal{I}_{2}\mathcal{I}_{2}W_{s2}\otimes\mathcal{I}^{T}_{r12}\mathcal{I}_{2}\mathcal{I}_{2}W_{r2})p \\
		=& (\mathcal{D}^{T}_{t12}\dfrac{\partial t}{\partial \theta}W_{t2}\otimes\mathcal{I}^{T}_{s12}W_{s2}\otimes\mathcal{I}^{T}_{r12}W_{r2})p
	\end{eqnarray}
\end{subequations}

\subsection{Opdiv}

The opdiv subroutine implements the divergence operation for a vector field $u$, defined on Mesh 1 (velocity mesh), integrated w.r.t the Mesh 2 test functions. We can represent $\partial r/\partial x, \partial s/\partial x, \ldots$ as matrices $D_{x}r, D_{x}s, \ldots$. For a one dimensional case, $D_{x}r$ is a diagonal matrix. Therefore, in one dimension, $D_{x} = D_{x}rD_{r}$ and $D^{T}_{x} = D^{T}_{r}(D_{x}r)^{T}$ \textit{etc.} 
\begin{eqnarray}
	 \mathlarger\int_{\Omega} q\nabla\cdot u  = & \mathlarger\int_{\Omega} q\dfrac{\partial }{\partial x_{i}}(u_{i}) d\Omega \nonumber \\
	 = & \mathlarger\int_{\Omega} q\dfrac{1}{\mathcal{J}}\dfrac{\partial r_{j}}{\partial x_{i}}\dfrac{\partial u_{i}}{\partial r_{j}} \mathcal{J} d\hat{\Omega} \nonumber \\
	 =& \mathlarger\sum_{k} q(x_{k})W(x_{k})\left(\dfrac{\partial r_{j}}{\partial x_{i}}\dfrac{\partial u_{i}}{\partial r_{j}}(x_{k})\right)  \nonumber \\
	 =& \mathlarger\sum_{k} q(x_{k})W(x_{k})(D_{x_{i}}r_{j})(D_{r_{j}}u_{i})  \nonumber \\
	 =&  q_{k}W_{k}(D_{x_{i}}r_{j})(D_{r_{j}}u_{i})  \nonumber
\end{eqnarray}
In the absence of cross geometric factors this becomes
\begin{eqnarray}
	 \mathlarger\int_{\Omega} q\nabla\cdot u  = &  q_{k}W_{k}(D_{x}rD_{r}u + D_{y}sD_{s}v + D_{z}tD_{t}w)  \nonumber \\
	 \mathlarger\int_{\Omega} q\nabla\cdot u  = & \left\{
	 \begin{split}
	    qW(\mathcal{I}_{t12}\otimes\mathcal{I}_{s12}\otimes D_{x}rD_{r})u \\ 
	    + qW(\mathcal{I}_{t12}\otimes D_{y}sD_{s}\otimes\mathcal{I}_{r12})v \\ 
	    + qW(D_{z}tD_{t}\otimes\mathcal{I}_{s12}\otimes \mathcal{I}_{r12})w
 	 \end{split}\right.  \\
%
	 \mathlarger\int_{\Omega} q\nabla\cdot u  = & \left\{
	 \begin{split}
	q(W_{t}\mathcal{I}_{t12}\otimes W_{s}\mathcal{I}_{s12}\otimes W_{r}D_{x}rD_{r})u \\ 
	+ q(W_{t}\mathcal{I}_{t12}\otimes W_{s}D_{y}sD_{s}\otimes W_{r}\mathcal{I}_{r12})v \\ 
	+ q(W_{t}D_{z}tD_{t}\otimes W_{s}\mathcal{I}_{s12}\otimes W_{r}\mathcal{I}_{r12})w
\end{split}\right.  
\end{eqnarray}
Here $D_{r},D_{s},D_{t}$ is essential $D_{r12},D_{s12},D_{t12}$, since we are evaluating the derivative of a Mesh 1 field on Mesh 2 points.

For the cylindrical case we have
\begin{eqnarray}
	\mathlarger\int_{\Omega} q\nabla\cdot u  = & \mathlarger\int_{\Omega} q\dfrac{\partial }{\partial x_{i}}(u_{i}) d\Omega \nonumber \\
	= & \mathlarger\int_{\Omega} q\dfrac{\partial u_{i}}{\partial x_{i}} R \mathcal{J} d\hat{\Omega} \nonumber \\
	= & \mathlarger\int_{\Omega} q\left(\dfrac{\partial u}{\partial x} + \dfrac{1}{R}\dfrac{\partial (Rv)}{\partial R} + \dfrac{1}{R}\dfrac{\partial w}{\partial \theta} \right) R \mathcal{J} d\hat{\Omega} \nonumber \\
	= & \mathlarger\int_{\Omega} q\left(\dfrac{\partial u}{\partial x} + \dfrac{\partial v}{\partial R} + \dfrac{v}{R}+ \dfrac{1}{R}\dfrac{\partial w}{\partial \theta} \right) R \mathcal{J} d\hat{\Omega} \nonumber \\
	= & \mathlarger\int_{\Omega} q\left(\dfrac{1}{\mathcal{J}}\dfrac{\partial r_{j}}{\partial x}\dfrac{\partial u}{\partial r_{j}} + \dfrac{1}{\mathcal{J}}\dfrac{\partial r_{j}}{\partial R}\dfrac{\partial v}{\partial r_{j}} + \dfrac{v}{R} + \dfrac{1}{R}\dfrac{1}{\mathcal{J}}\dfrac{\partial r_{j}}{\partial \theta}\dfrac{\partial w}{\partial r_{j}} \right) R \mathcal{J} d\hat{\Omega} \nonumber \\
	= & \mathlarger\int_{\Omega} q\left(\dfrac{\partial r_{j}}{\partial x}\dfrac{\partial u}{\partial r_{j}} + \dfrac{\partial r_{j}}{\partial R}\dfrac{\partial v}{\partial r_{j}} + \dfrac{\mathcal{J}v}{R} + \dfrac{1}{R}\dfrac{\partial r_{j}}{\partial \theta}\dfrac{\partial w}{\partial r_{j}} \right) R  d\hat{\Omega} \nonumber \\	
	= &  W_{k}q_{k}\left(R\dfrac{\partial r_{j}}{\partial x}\dfrac{\partial u}{\partial r_{j}} + R\dfrac{\partial r_{j}}{\partial R}\dfrac{\partial v}{\partial r_{j}} + \dfrac{R\mathcal{J}v}{R} + \dfrac{\partial r_{j}}{\partial \theta}\dfrac{\partial w}{\partial r_{j}} \right) 
\end{eqnarray}
For undeformed elements, $\mathcal{J}$ is a constant throughout the element. Therefore in kronecker notation we have
\begin{eqnarray}
	\mathlarger\int_{\Omega} q\nabla\cdot u  = \left\{
		\begin{split}
			Wq(\mathcal{I}_{12}\otimes R\otimes D_{x12}rD_{r})u \\
			+ Wq(\mathcal{I}_{12}\otimes (RD_{R12}sD_{s} + \mathcal{J}\mathcal{I}_{12})\otimes \mathcal{I}_{12})v \\
			+ Wq(D_{\theta12}tD_{t}\otimes \mathcal{I}_{12}\otimes \mathcal{I}_{12})w
		\end{split}\right. \\
	\mathlarger\int_{\Omega} q\nabla\cdot u  = \left\{
\begin{split}
	q(W_{t}\mathcal{I}_{12}\otimes W_{s}R\otimes W_{r}D_{x12}rD_{r})u \\
	+ q(W_{t}\mathcal{I}_{12}\otimes W_{s}(RD_{R12}sD_{s} + \mathcal{J}\mathcal{I}_{12})\otimes W_{r}\mathcal{I}_{12})v \\
	+ q(W_{t}D_{\theta12}tD_{t}\otimes W_{s}\mathcal{I}_{12}\otimes W_{r}\mathcal{I}_{12})w
\end{split}\right.	
\end{eqnarray}

\subsection{Pressure Pseudo-Laplacian}

\begin{eqnarray}
	S_{\Delta t} = D QD^{T} \label{eqn:pressure_pseudo_laplacian}
\end{eqnarray}
where, if $Q=H^{-1}$ there is no decoupling error and we have the Uzawa algorithm. Alternately, $Q=B^{-1}$ in which case we incur a decoupling error but avoid the nested iterations since $B$, being diagonal, can be trivially inverted. 


%\begin{figure}[h]
%	\centering
%	\includegraphics[width=0.75\linewidth]{cylinder_surf}
%	\caption{FSI surface at two different time instants}
%	\label{fig:fsi_surf}
%\end{figure}



\FloatBarrier

\bibliographystyle{elsarticle-harv}
\bibliography{cylindrical}
%\printbibliography[notcategory=ignore]
\end{document}


